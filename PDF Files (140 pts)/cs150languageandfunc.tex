\documentclass[11pt,a4paper]{report}
%\renewcommand{\rmdefault}{phv} % Arial
%\renewcommand{\sfdefault}{phv} % Arial
\usepackage{helvet}
\renewcommand{\familydefault}{\sfdefault}
\setlength{\parskip}{1em}
\usepackage{graphicx}
\usepackage{titlepic}
\usepackage{listings}
\usepackage{color}

\definecolor{dkgreen}{rgb}{0,0.6,0}
\definecolor{gray}{rgb}{0.5,0.5,0.5}
\definecolor{mauve}{rgb}{0.58,0,0.82}

\lstset{frame=tb,
  aboveskip=3mm,
  belowskip=3mm,
  showstringspaces=false,
  columns=flexible,
  basicstyle={\small\ttfamily},
  numbers=none,
  numberstyle=\tiny\color{gray},
  keywordstyle=\color{blue},
  commentstyle=\color{dkgreen},
  stringstyle=\color{mauve},
  breaklines=true,
  breakatwhitespace=true,
  tabsize=3
}

\title{\includegraphics[scale=1.5]{twt}\\~\\~\\ \LARGE{\textbf{TWT Programming
Language}}\\~\\ \large{(Talagang Walang Tulog)}\\~\\ \large{Language Design and Functionalities}}
\date{~\\ December 2015}
\author{Jahziel Rae Arceo\\ Gabriel Kelly Navarro\\ Mikaela Jun Lenon\\ Maria
Rosario Gueco\\ Marbille Juntado\\~\\~\\ CS 150 HTWX}

\begin{document}

\maketitle

\chapter{Introduction}

\section{The TWT programming language}

TWT is simply a programming language designed out of curiosity. There were
really many challenges of deciding on which fits best on the category of a
usable language, which the designers took so much time for considering and
creating such.

\subsection{Language Name}

Halfway before the appropriate parser was created for the
language, there was no appropriate meaning that the programmers can imply on
the initials TWT, except that it was Twitter-inspired.
However, since they were inspired by the countless nights that they weren't of
good sleep, they made a way for the meaning of the name. TWT refers to "Talagang Walang Tulog",
or "really no sleep", the phenomenon that happened to them during the creation
of the language.

\subsection{Paradigm}

The TWT programming language is imperative and functional. The programming
language works in a fashion nearly similar to C, Python, and Pascal. Function
declarations are C-styled. Also, keywords that begin and end a program are used
in the language, which seems to be like Pascal. The language also borrows features from Python, which was the
language used to create the new language.

\subsection{Inspiration}

Twitter, a social networking site, was the main inspiration on creating the
language. As seen on the next chapter, the keywords and reserved words of the
language mostly came from the jargons seen on the website. \par

Other than Twitter, the programmers also made the intricacy of programming an
inspiration to create the language. Syntaxes on the language are simpler
compared to other languages, which makes new programmers learn the language
easily.

\chapter{Grammar Definition}

In this chapter, we get to learn about the grammar of the language using its
Backus-Naur form (BNF). By this, we can get a hint of what the language looks
like on encoding.

\section{Identifiers}

The following conventions apply to variables and values entered in the
programming language:

\begin{lstlisting}
<INT> ->				<Digit><INT> | <Digit>
<Digit> ->			"0" | "1" | "2" | "3" | "4" | "5" | "6"
						| "7" | "8" | "9"

<CHAR> ->			"'"<ASCII>"'"

<FLOAT> ->			<Digit>"."<Digit>
						(Note: Strictly followed)

<STRING> -> 		"""<STRINGEXT>"""

<STRINGEXT> -> 	<CHAR><STRINGEXT> | <CHAR> | EPSILON

<VARIABLE> -> 		<ALPHABET><VARIABLEEXT>

<VARIABLEEXT> -> <ALPHANUMERIC><VARIABLEEXT>
						| <ALPHANUMERIC> | EPSILON
\end{lstlisting}

Note that the grammar for float type values is strict in the programming
language, so it will not accept an integer as float. The data must explicitly
indicate that it is indeed a floating-point value.

\section{Data Types}

The TWT language is statically-typed. The programmer should explicitly state the
data type he is to use.

Data types were defined in the language grammar as follows:

\begin{lstlisting}
<Dtype> -> "@INT" | "@CHIRP" | "@COKE" | "@MSG" | "@TRALSE"
\end{lstlisting}

Data types are as follows:
\begin{itemize}
  \item \textbf{@INT:} integer
  \item \textbf{@CHIRP:} character (ASCII)
  \item \textbf{@COKE:} float
  \item \textbf{@MSG:} string
  \item \textbf{@TRALSE:} boolean
\end{itemize}

The '@' symbol is used to easily identify words as types. By inspiration, it
takes the @-mention paradigm of Twitter as its data type recognition
mechanism.

\section{Expressions and Assignment Statements}

The following rules apply to expressions and assignments:
\begin{lstlisting}
<Assignment> -> 	<DType> VARIABLE "=" VARIABLE
                	| <DType> VARIABLE "=" <Reading>
                	| "@INT" VARIABLE "=" INT
                	| "@CHIRP" VARIABLE "=" CHAR
                	| "@COKE" VARIABLE "=" FLOAT
                	| "@MSG" VARIABLE "=" STRING
                	| "@TRALSE" VARIABLE "=" "YES"
                	| "@TRALSE" VARIABLE "=" "NO"
                	| <DType> VARIABLE "=" "(" <Exp> ")"

<Reading> -> 		"REPLY" VARIABLE

<Exp> -> 			<Term> <ExpPrime>

<ExpPrime> ->		"+" <Term> <EspPrime>
            		| "-" <Term> <ExpPrime>
            
<Term> -> 			<Fact> <TermPrime>

<TermPrime> ->		"*" <Fact> <TermPrime>
            		| "/" <Fact> <TermPrime>
            
<Fact> -> 			INT | CHAR | FLOAT | VARIABLE | "(" <Exp> ")"

\end{lstlisting}

\section{Statement Level Control Structures}

The following rules apply to control structures available in the language:
\begin{lstlisting}
<State> ->			<Loop> "#" <StatePrime>
        				| <If> "#" <StatePrime>
        				| <Assignment> "#" <StatePrime>
        				| <Call> "#" <StatePrime>
        				| <Printing> "#" <StatePrime>
        				| <Control> "#" <StatePrime>
        			
<StatePrime> -> 	<Loop> "#" <StatePrime>
        				| <If> "#" <StatePrime>
        				| <Assignment> "#" <StatePrime>
        				| <Call> "#" <StatePrime>
        				| <Printing> "#" <StatePrime>
        				| <Read> "#" <StatePrime>
        				| <Control> "#" <StatePrime>
        			
<Loop> -> 			"RT" <Condition> {" <Block> "}"

<If> -> 				"IF" <Condition> "FOLLOW" "{" <Block> "}"
						<ElseIf>
						<Else>
        				| "IF" <Condition> "FOLLOW" "{" <Block> "}"
        				<Else>
        				| "IF" <Condition> "FOLLOW" "{" <Block> "}"
        			
<ElseIf> ->			"ELSEIF" <Condition> "FOLLOW" "{" <Block> "}"
						<Elseif>

<Else> -> 			"ELSE" <Condition> "FOLLOW" "{" <Block> "}"
\end{lstlisting}

\section{Subprograms}

This is how functions or subprograms are declared in the language:
\begin{lstlisting}
<Declaration> ->	<Dtype> VARIABLE "(" <Args> ")"
						"{" <Block> <Return> "}"
						| <Dtype> VARIABLE "(" <Args> ")" "{" <Block> "}"
\end{lstlisting}

\section{Other syntaxes not listed on previous sections}

\begin{lstlisting}
<Program> ->		<Declaration> <Main>
            		| <Main>
            
<Main> ->			"LOGIN" <Block> "LOGOUT"
          			| "LOGIN" "LOGOUT"
          			
<Block> ->			<State>

<Call> ->			"HOOT "VARIABLE "(" <Args> ")"

<Printing> -> 		"TWEET" VARIABLE
              		| "TWEET" "(" <EXP> ")"
              		| "TWEET" INT
              		| "TWEET" FLOAT
              		| "TWEET" CHAR
              		| "TWEET" STRING
              		| "TWEET" TRUE
              		| "TWEET" FALSE
<Reading> ->		"REPLY" VARIABLE

<Control> ->		"UNFOLLOW" | "LIKE" | "BLOCK"

<Condition> ->		"(" <Conditional> ")" | "~" <Condition>

<Conditional> ->	VARIABLE <Conditional> VARIABLE
          			| { VARIABLE | INT | CHAR | FLOAT } <CondOp> <Condition>
          			| <Conditional> <CondOp> <Conditional>
          			| <Condition> <CondOp> { VARIABLE | INT | CHAR | FLOAT }
          			| { VARIABLE | INT | CHAR | FLOAT } <CondOp>
          			{ VARIABLE | INT | CHAR | FLOAT }
          			
<CondOp> ->			">=" | "<=" | "==" | ">" | "<"

<Args> ->			<Dtype> <Vname> "," <Args>

<Return> ->			"REPORT" { INT | CHAR | FLOAT | STRING | VARIABLE }
						| "REPORT" <Exp>
\end{lstlisting}

\chapter{Lexical and Syntax Analysis}

\section{Parser}

The parser for the language uses LR parsing, so no parse tables were used to
parse the strings for the language. By that, it is obvious in the grammar that
revisions for left recursion were done, so that no conflict will happen on the
parsing process.

\chapter{Names, Binding, and Scoping}

\section{Case sensitivity}

The TWT programming language was designed in a way that the user must
exactly write the language in whichever case it is. By that, TWT is
case-sensitive. For example, \textbf{@int} is not the same with \textbf{@INT}.

\section{Reserved words}

The following is a list of the reserved words in the language:

\begin{center}
\begin{tabular}{ c c c c c }
 LOGIN & LOGOUT & REPORT & TWEET & IF \\ 
 ELSEIF & ELSE & HOOT & REPLY & UNFOLLOW \\  
 @INT & @CHIRP & @COKE & @MSG & @TRALSE \\
 LIKE & BLOCK & YES & NO
\end{tabular}
\end{center}

\section{Name form}

By convention, as explained in the grammar, the names must be alphabetic at the
start of each string, for variables. The language follows C-style rules.

\section{Binding}

The language follows Python-based binding, but on the programming language
scope, it was designed to be statically bound. As noticed, Python does dynamic
binding, however, the programmers chose to require having types specified before
each run.

\section{Lifetime and scope}

Conventions were also followed on variable lifetime and scope. That is, what
Python generally has on variable lifetime and scope applies to TWT as well.

\section{Blocks}

In TWT, blocks can be determined by curly braces, as what C has. Note the
following code example:

\begin{lstlisting}
RT ( X < 4 ) { 
		X = ( X + 1 ) #
		Y = ( Y + 2 ) #
		TWEET X #
 } #
\end{lstlisting}

The example above is a loop. It has the three statements "X = ( X + 1 ) \#", "Y
= ( Y + 2 ) \#", and "TWEET X \#" in the block example.

\chapter{Data types}

\section{Primitive data types}

As referenced to the grammar in Chapter 2, these are the data types available in
the language:

\begin{itemize}
  \item \textbf{@INT:} integer
  \item \textbf{@CHIRP:} character (ASCII)
  \item \textbf{@COKE:} float
  \item \textbf{@MSG:} string
  \item \textbf{@TRALSE:} boolean
\end{itemize}

\section{Strings}

The programming language supports strings, and is a primitive data type. Like C,
the strings are determined by double quotation marks.\par

There are no limitations on string declarations, and everything that applies to
every other language also applies to TWT.

\section{Note on user-defined types and type-checking}

As of writing, user-defined typing is not available in the language. The user
must strictly use the available data types.\par

On type-checking and coercion, Python conventions are used, that is, the
features that Python has are used in the language.

\chapter{Expression and Assignment Statements}

\end{document}
